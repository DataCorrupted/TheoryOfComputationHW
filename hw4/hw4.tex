\documentclass{article}

\usepackage{fancyhdr}
\usepackage{extramarks}
\usepackage{amsmath}
\usepackage{amsthm}
\usepackage{tikz}
\usepackage{enumitem}
\usepackage{comment}

\usetikzlibrary{automata, positioning}

\topmargin=-0.45in
\evensidemargin=0in
\oddsidemargin=0in
\textwidth=6.5in
\textheight=9.0in
\headsep=0.25in

\linespread{1.1}

\pagestyle{fancy}
\lhead{\hmwkAuthorName\ -\ \hmwkAuthorID}
\chead{\hmwkClass: Homework \hmwkNo}
\rhead{\firstxmark}
\lfoot{\lastxmark}
\cfoot{\thepage}

\renewcommand\headrulewidth{0.4pt}
\renewcommand\footrulewidth{0.4pt}

\newcommand{\enterProblemHeader}[1]{
    \nobreak\extramarks{}{Problem \arabic{#1} continued on next page\ldots}\nobreak{}
    \nobreak\extramarks{Problem \arabic{#1} (continued)}{Problem \arabic{#1} continued on next page\ldots}\nobreak{}
}

\newcommand{\exitProblemHeader}[1]{
    \nobreak\extramarks{Problem \arabic{#1} (continued)}{Problem \arabic{#1} continued on next page\ldots}\nobreak{}
    \stepcounter{#1}
    \nobreak\extramarks{Problem \arabic{#1}}{}\nobreak{}
}

\setcounter{secnumdepth}{0}
\newcounter{homeworkProblemCounter}
\setcounter{homeworkProblemCounter}{1}
\nobreak\extramarks{Problem \arabic{homeworkProblemCounter}}{}\nobreak{}

\newenvironment{homeworkProblem}[1][-1]{
    \ifnum#1>0
        \setcounter{homeworkProblemCounter}{#1}
    \fi
    \section{Problem \arabic{homeworkProblemCounter}}
    \enterProblemHeader{homeworkProblemCounter}
}{
    \exitProblemHeader{homeworkProblemCounter}
}

\newenvironment{solution}{
    \subsection{Solution}
}

\newcommand{\hmwkNo}{4}
\newcommand{\hmwkDueDate}{Saturday, June 8, 2019 at 11:59pm}
\newcommand{\hmwkClass}{CS244 Theory of Computation}
\newcommand{\hmwkClassInstructor}{Fu Song}
\newcommand{\hmwkAuthorName}{Yuyang Rong}
\newcommand{\hmwkAuthorID}{69850764}


\newcommand{\TM}{\mathsf{TM}}


\title{
    \vspace{-0.4in}
    \textmd{\textbf{\hmwkClass \\ Homework \hmwkNo}}\\
    \normalsize\vspace{0.1in}\small{Due: \hmwkDueDate}\\
}

\author{\hmwkAuthorName\ -\ \hmwkAuthorID}
\date{}

\begin{document}

\maketitle
\thispagestyle{fancy}

You may discuss this assignment with other students and work
on the problems together. However, your write-up should be your own individual work and you should indicate in your submission who you worked with, if applicable. You should use the {\LaTeX} template provided by us to write your solution and submit the generated PDF file into Gradescope.

Note: You only need to submit your solutions to the \textbf{\emph{first three}} problems. The other problems are optional. \\

I worked with: (Name, ID), (Name, ID), \ldots

\begin{homeworkProblem}
\begin{enumerate}[label=(\alph*)]
    \item Show that $A_\textsf{LBA} = \{ \langle B, w \rangle \mid B$ is an \textsf{LBA} that accepts input $w \}$ is PSPACE-complete.
    \item Show that $E_\textsf{DFA} = \{ \langle A \rangle \mid A$ is a \textsf{DFA} and $L(A) = \emptyset \}$ is NL-complete.
\end{enumerate}
\end{homeworkProblem}
\begin{solution}\begin{enumerate}[label=(\alph*)]
    \item \begin{proof}
        It is trivial that $A_\textsf{LBA} \in$ PSPACE since it's linear bounded.

    \end{proof}

    \item \begin{proof}
        We begin by prooving that $E_\textsf{DFA} \in $ NL.
        M on input $\langle A \rangle$:
        \begin{enumerate}
            \item Place the start node $s$ in the workspace
            \item Repeat the following at most $|Q|$ steps:
            \begin{enumerate}
                \item Nondeterministically chooses a next node n
                \item Replace current node on the tape with n.
            \end{enumerate}
            \item if terminal state is reached, accept; else reject.
        \end{enumerate}
        Then only $O(log |Q|)$ space is used to store the number of the steps on the tape, $E_\textsf{DFA} \in $ NL.

        Now let's reduce  $E_\textsf{DFA}$ to $PATH$ in polynomial time.
            
    \end{proof}
\end{enumerate}\end{solution}

\begin{homeworkProblem}
Describe a deterministic, polynomial-time \textit{SAT}-oracle Turing machine $M^\textit{SAT}$ that takes as input a directed graph $G$ and nodes $s$ and $t$, and outputs a Hamiltonian path from $s$ to $t$ if one exists. If none exist, then $M^\textit{SAT}$ outputs \textbf{No Hamiltonian path}.
\end{homeworkProblem}
\begin{solution}
    We reduce SAT problem to Hamiltonian path, then use the oracle to tell the answer and convert it back to our path problem. If the SAT cannot be satisfied, then there is no such path.

    For each edge, label it with $e_{ij}$, standing for an edge from vertex $i$ to vertex $j$.
    In our SAT, each edge stands for a variable, therefore we have $x_{ij}$, it is true if we take edge $e_{ij}$ in our path.
    Now for each vertex, call it vertex $k$.
    
    Suppose $k$ has $p$ outward edges $e_{ko_1}, e_{ko_2}, \cdots e_{ko_p}$ write the following:
    $$ 
        \phi_{out}(k) = (x_{ko_1} \wedge \overline{x_{ko_2}} \wedge \cdots \wedge \overline{x_{ko_p}}) 
                \lor (x_{ko_2} \wedge \overline{x_{ko_1}} \wedge \cdots \wedge \overline{x_{ko_p}})
                \lor \cdots 
                \lor (x_{ko_p} \wedge \overline{x_{ko_1}} \wedge \cdots \wedge \overline{x_{ko_{p-1}}})
    $$

    Suppose $k$ has $q$ inward edges $e_{ki_1}, e_{ki_2}, \cdots e_{ki_q}$ write the following:    
    $$ 
        \phi_{in}(k) = (x_{ki_1} \wedge \overline{x_{ki_2}} \wedge \cdots \wedge \overline{x_{ki_q}}) 
                \lor (x_{ki_2} \wedge \overline{x_{ki_1}} \wedge \cdots \wedge \overline{x_{ki_q}})
                \lor \cdots 
                \lor (x_{ki_p} \wedge \overline{x_{ki_1}} \wedge \cdots \wedge \overline{x_{ki_{q-1}}}) 
    $$
    $$  \phi(k) = \phi_{in}(k) \wedge \phi_{out}(k) $$
            
    Explained: $\phi_{in}(k)$ guarantees that vertex k has only one edge in, $\phi_{out}(k)$ guarantees that there is only one edge out.

    Label every vertex except $s$ and $t$ from $1$ to $n$, then solving 
        $\phi = \phi_{out}(s) \wedge \phi_{in}(t) \wedge \phi(1) \wedge \phi(2) \wedge \cdots \wedge \phi(n) $

    \textbf{Proof} that an answer to $\phi$ solves Hamiltonian Path:
    Since $x_{ij}$ means we take $e_{ij}$ and for each vertex $k$, $\phi(k)$ being true guarantees that vertex has one and only one edge in and out.
    $s$ only needs out edge and $t$ only nees in edge.
    That means using $x_{ij}$ we can construct a path from $s$ to $t$ that covers every other node.

    Construct a $\TM$ on input:
    \begin{enumerate}
        \item Run oracle on input $\phi$ as described above to solve SAT problem.
        \item If oracle returns false, that $\phi$ is not satisfiable, output \textbf{No Hamiltonian path} and reject.
        \item Or else according to the answer to generate a path and accept.
    \end{enumerate}
\end{solution}

\begin{homeworkProblem}
Say that a probabilistic algorithm \textbf{\emph{uses randomness r(n)}} if it uses at most $r(n)$ coin tosses on each computation thread.
\begin{enumerate}[label=(\alph*)]
    \item Recall the probabilistic algorithm for $\textit{EQ}_\textsf{ROBP}$ we presented. How much randomness does it use when it is run on two branching programs that have $m$ input variables? Give your answer as a function of $m$ using big-O notation. Explain your reasoning.
    \item Let $\mathrm{BPP}[f(n)] = \{A \mid A$ is decided by a probabilistic, polynomial time \textsf{TM} that uses at most $O(f(n))$ randomness on all inputs of length $n\}$. Show that $\mathrm{BPP}[\log(n)] \subseteq \mathrm{P}$.
\end{enumerate}
\end{homeworkProblem}
\begin{solution}\begin{enumerate}[label=(\alph*)]
    \item $r(m) = O(2^m)$
    \item \begin{proof}
        We are going to proof this by \emph{hack} the random function and make it a deterministic one yet still solve our problem.

        $\forall a \in BPP[log(n)], \exists M_a$ that uses at most $O(log(n))$ randomness.
        Construct the following $\TM$ M:
        \begin{enumerate}
            \item For i = 0 .. n repeat the following:
            \item \begin{enumerate}
                \item Use $i$'s binary form to decide the output of a random function. If k-th binary of $i$ is 0 then random function deterministicly outputs 0 on k-th coin toss, or else it outputs 1.
                \item Run $M_a$ on input $a$ with out \emph{hacked} random function.
                \item If $M_a$ accepts, accept, else continue;
            \end{enumerate}
            \item reject
        \end{enumerate}
        
        Proof that our construction works:

        Since $M_a$ that uses at most $O(log(n))$ randomness, there are at most $n$ out comes for $log(n)$ coin flits, in $M$, all these possible out comes have been tried to decide if $M_a$ accepts. Therefore our new $\TM$ still reflects $M_a$'s behavior.

        Since $M_a$ runs in polynomial time, we repeated it $n$ times, so $M$ still runs in polynomial time.
    \end{proof}
\end{enumerate}\end{solution}

\begin{homeworkProblem}
(Optional) For any positive integer $x$, let $x^\mathcal{R}$ be the integer whose binary representation is the reverse of the binary representation of $x$. (Assume no leading $\mathsf{0}$s in the binary representation of $x$.) Define the function $\mathcal{R}^+ : \mathcal{N} \rightarrow \mathcal{N}$ where $\mathcal{R}^+(x) = x + x^\mathcal{R}$.
\begin{enumerate}[label=(\alph*)]
    \item Let $A_2 = \{ \langle x, y \rangle \mid \mathcal{R}^+(x) = y \}$. Show $A_2 \in \mathrm{L}$.
    \item Let $A_3 = \{ \langle x, y \rangle \mid \mathcal{R}^+(\mathcal{R}^+(x)) = y \}$. Show $A_3 \in \mathrm{L}$.
\end{enumerate}
\end{homeworkProblem}
\begin{comment}
\begin{solution}\begin{enumerate}
\item  $\TM_{A_2}$ on input $\langle x, y \rangle$
    \begin{enumerate}
        \item 
    \end{enumerate}
\item 
\end{enumerate}\end{solution}
\end{comment}

\begin{homeworkProblem}
(Optional) For branching program $B$ and $w = w_1 \dotso w_m$, where each $w_i \in \{\mathsf{0}, \mathsf{1}\}$, let $B(w)$ be the output of $B$ when its input variables $x_1, \dotsc, x_m$ are set $x_i = w_i$ for each $i$.
\begin{enumerate}[label=(\alph*)]
    \item Let $\mathit{ALL}_\mathsf{ROBP} = \{ \langle B \rangle \mid B$ is a read-once branching program and $B(w) = 1$ on all $w\}$. Show that $\mathit{ALL}_\mathsf{ROBP} \in \mathrm{P}$.
    \item Let $\mathit{ALL}_\mathsf{BP} = \{ \langle B \rangle \mid B$ is a branching program and $B(w) = 1$ on all $w\}$. Show that $\mathit{ALL}_\mathsf{BP}$ is coNP-complete.
\end{enumerate}
\end{homeworkProblem}
\begin{solution}\begin{enumerate}
\item \begin{proof}
    Construct a $\TM$ M:
    \begin{enumerate}
        \item for i = 1..m
        \begin{enumerate}
            \item set $w_i$ to 1 and see if $B(w)$ can evaluate to true.
            \item set $w_i$ to 0 and see if $B(w)$ can evaluate to true.
            \item The two test above must happen withnot any knowledge of other variables. For example, using short circuit evaluation, we can know $x | y$ is true if x is true, regardless of y's value.
            \item If one test fails, reject.
        \end{enumerate}
        \item accept.
    \end{enumerate}
\end{proof}
\item \begin{proof}
\end{proof}
\end{enumerate}\end{solution}

\begin{homeworkProblem}
(Optional) Prove that if $A \subseteq \{\mathsf{0}, \mathsf{1}\}^*$ is a regular language, a family of branching programs $(B_1, B_2, \dotsc)$ exists where each $B_n$ accepts exactly the strings in $A$ of length $n$ and is bounded in size by a constant times $n$.
\end{homeworkProblem}

\end{document}