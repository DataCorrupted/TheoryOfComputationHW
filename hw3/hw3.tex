\documentclass{article}

\usepackage{fancyhdr}
\usepackage{extramarks}
\usepackage{amsmath}
\usepackage{amsthm}
\usepackage{tikz}
\usepackage{enumerate}

\usetikzlibrary{automata, positioning}

\topmargin=-0.45in
\evensidemargin=0in
\oddsidemargin=0in
\textwidth=6.5in
\textheight=9.0in
\headsep=0.25in

\linespread{1.1}

\pagestyle{fancy}
\lhead{\hmwkAuthorName\ -\ \hmwkAuthorID}
\chead{\hmwkClass: Homework \hmwkNo}
\rhead{\firstxmark}
\lfoot{\lastxmark}
\cfoot{\thepage}

\renewcommand\headrulewidth{0.4pt}
\renewcommand\footrulewidth{0.4pt}

\newcommand{\enterProblemHeader}[1]{
    \nobreak\extramarks{}{Problem \arabic{#1} continued on next page\ldots}\nobreak{}
    \nobreak\extramarks{Problem \arabic{#1} (continued)}{Problem \arabic{#1} continued on next page\ldots}\nobreak{}
}

\newcommand{\exitProblemHeader}[1]{
    \nobreak\extramarks{Problem \arabic{#1} (continued)}{Problem \arabic{#1} continued on next page\ldots}\nobreak{}
    \stepcounter{#1}
    \nobreak\extramarks{Problem \arabic{#1}}{}\nobreak{}
}

\setcounter{secnumdepth}{0}
\newcounter{homeworkProblemCounter}
\setcounter{homeworkProblemCounter}{1}
\nobreak\extramarks{Problem \arabic{homeworkProblemCounter}}{}\nobreak{}

\newenvironment{homeworkProblem}[1][-1]{
    \ifnum#1>0
        \setcounter{homeworkProblemCounter}{#1}
    \fi
    \section{Problem \arabic{homeworkProblemCounter}}
    \enterProblemHeader{homeworkProblemCounter}
}{
    \exitProblemHeader{homeworkProblemCounter}
}

\newenvironment{solution}{
    \subsection{Solution}
}

\newcommand{\hmwkNo}{3}
\newcommand{\hmwkDueDate}{Wednesday, May 29, 2019 at 11:59pm}
\newcommand{\hmwkClass}{CS244 Theory of Computation}
\newcommand{\hmwkClassInstructor}{Fu Song}
\newcommand{\hmwkAuthorName}{Yuyang Rong}
\newcommand{\hmwkAuthorID}{69850764}

\title{
    \vspace{-0.4in}
    \textmd{\textbf{\hmwkClass \\ Homework \hmwkNo}}\\
    \normalsize\vspace{0.1in}\small{Due: \hmwkDueDate}\\
}

\author{\hmwkAuthorName\ -\ \hmwkAuthorID}
\date{}

\newcommand{\NP}{\mathrm{NP}}
\renewcommand{\P}{\mathrm{P}}

\begin{document}

\maketitle
\thispagestyle{fancy}

You may discuss this assignment with other students and work
on the problems together. However, your write-up should be your own individual work and you should indicate in your submission who you worked with, if applicable. You should use the {\LaTeX} template provided by us to write your solution and submit the generated PDF file into Gradescope.

Note: You only need to submit your solutions to the \textbf{\emph{first three}} problems. The other problems are optional. \\

I worked with: (Name, ID), (Name, ID), \ldots

\begin{homeworkProblem}
The \textbf{\emph{Kolmogorov complexity}} of a bit string $b$, $K_L(b)$, is the length of the shortest program in language $L$ that outputs $b$ and only $b$. Is $K_L(b)$ computable? Prove your answer.
\end{homeworkProblem}
\begin{solution}
    
\end{solution}

\begin{homeworkProblem}
Let $\textit{SHUFFLE} = \{ \langle w, x, y \rangle \mid w = a_1 b_1 \dotso a_k b_k$ for $k \ge 0$ where $x = a_1 a_2 \dotso a_k$ and $y = b_1 b_2 \dotso b_k$, each $a_i, b_i \in \Sigma^* \}$.
\begin{enumerate}[(a)]
    \item Show that $\textit{SHUFFLE} \in \NP$.
    \item Show that $\textit{SHUFFLE} \in \P$.
\end{enumerate}
\end{homeworkProblem}
\begin{solution}
    Since $\NP \subset \P$, simply proofing that $\textit{SHUFFLE} \in \P$ is suffice.
    Below, we would proof that $\textit{SHUFFLE} \in \P$.

\begin{proof}

\end{proof}\end{solution}

\begin{homeworkProblem}
Let $\textit{SET-SPLITTING} = \{ \langle S, C \rangle \mid S$ is a finite set and $C = \{ C_1, \dotsc, C_k \}$ is a collection of subsets of $S$, where the elements of $S$ can be colored red or blue so every $C_i$ has at least one red element and at least one blue element$ \}$. Show that $\textit{SET-SPLITTING}$ is $\NP$-complete.
\end{homeworkProblem}

\begin{solution}\begin{proof}
    We will proof that $\textit{SET-SPLITTING}$ is $\NP$-complete by reduce it to SAT.

    Suppose $S = \{s_1, s_2, \cdots, s_n\}$, we construct a boolean variable with $x_1, x_2, \cdots, x_n$, where $x_k$ is true if $s_k$ is colored blue and false if red.

    Then for each $C_k = \{s_{c_{k_1}}, s_{c_{k_2}}, \cdots, s_{c_{k_m}}\}$, we can construct a clause: $$\phi_{k_{Red}} = (x_{c_{k_1}} \lor x_{c_{k_2}} \lor \cdots \lor x_{c_{k_m}})$$
    $$\phi_{k_{Blue}} = (\overline{x_{c_{k_1}}} \lor \overline{x_{c_{k_2}}}, \lor \cdots \overline{x_{c_{k_m}}})$$
    $$\phi_k = \phi_{k_{Red}} \wedge \phi_{k_{Blue}}$$

    Finally we have a boolean expression: $$\phi = \phi_1 \wedge \phi_2 \wedge \cdots \wedge \phi_n$$.
    This reduction can happen in polynomial time.
    Now we are going to proof that $\textit{SET-SPLITTING} \Leftrightarrow \textit{SAT}$ using $\phi$.
    
    $\leftarrow$: If $\textit{SET-SPLITTING}$ is satisfied, $\forall k, \exists p, q$ such that $s_{c_{k_p}} \in C_k$ is red and  $s_{c_{k_q}} \in C_k$ is blue, meaning that $x_{c_{k_p}}$ is true and $x_{c_{k_q}}$ is false, $\overline{x_{c_{k_q}}}$ is true.

    Thus, $\forall k, \phi_{k_{Red}}$ and $\phi_{k_{Blue}}$ eva`luates to true, therefore $\phi$ evaluates to true. Using the value of $x_1, x_2, \cdots, x_n$, $\textit{SAT}$ problem can be solved.

    $\rightarrow$: Likewise, since $x_k$ and the color of $s_k$ is corresponded, if $\textit{SAT}$ is solved, using the value of the $x_k$ can tell us how to color $s_k$ and solve $\textit{SET\_SPLITTING}$.

    In the end, since we can reduce $\textit{SET\_SPLITTING}$ to $\textit{SAT}$ in polynomial time, $\textit{SAT}$ is $\NP$-complete, it is safe to claim that $\textit{SET\_SPLITTING}$ is $\NP$-complete too.
\end{proof}\end{solution}

% \begin{homeworkProblem}
% (Optional) Let $\textit{MODEXP} = \{ \langle a, b, c, p \rangle \mid a, b, c, p$ are positive binary integers such that $a^b \equiv c \pmod{p} \}$. Show that $\textit{MODEXP} \in \P$.
% \end{homeworkProblem}

\begin{homeworkProblem}
(Optional) Show that if $\P = \NP$, then every language $A \in \P$, except $A = \emptyset$ and $A = \Sigma^*$, is $\NP$-complete.
\end{homeworkProblem}
\begin{solution}\begin{proof}
    $\forall A \in \P$, it can be solved with answer $a$ in polynomial time.
    Then $\forall B$, we can decide the input for $A$ based on the output of $B$.
    If $B$ returns false, feed $A$ with any input except $a$, else feed in $a$.
    This way, we reduced $B$ to $A$ in polynomial time since $A, B \in \P$ because $\P = \NP$.
    Now again, since $\P = \NP$, $A \in \NP$, and with a polynomial reduction, we can claim that $\forall A \in P$, $A$ is $\NP$-complete.
\end{proof}\end{solution}

%\begin{homeworkProblem}
%(Optional) Show that if $\P = \NP$, a polynomial time algorithm exists that produces a satisfying assignment when given a satisfiable Boolean formula. (Hint: Use the satisfiability tester repeatedly to find the assignment bit-by-bit.)
%\end{homeworkProblem}
%\begin{solution}

\end{solution}


% \begin{homeworkProblem}
% (Optional)
% \begin{enumerate}[(a)]
%     \item Explain why the following argument fails to show that $\textit{MIN-FORMULA} \in \mathrm{coNP}$:
%     \begin{enumerate}[i.]
%         \item If $\phi \notin \textit{MIN-FORMULA}$, then $\phi$ has a smaller equivalent formula.
%         \item An $\mathsf{NTM}$ can verify that $\phi \in \overline{\textit{MIN-FORMULA}}$ by guessing that formula.
%     \end{enumerate}
%     \item Show (despite part a) that if $\P = \NP$, then $\textit{MIN-FORMULA} \in \P$.
% \end{enumerate}
% \end{homeworkProblem}

\end{document}